\documentclass[]{article}
\usepackage[margin=2.0cm]{geometry}
\usepackage{enumitem}
\usepackage{yhmath}
\usepackage{amsmath}
\usepackage{amssymb}
\usepackage{gensymb}
\usepackage{hyperref}
\usepackage{graphicx}
\usepackage{physics}
\usepackage{mathtools}
\usepackage{scalerel}
\usepackage{caption}
\usepackage{subfig}
\usepackage{bm}

\DeclareMathOperator*{\Bigcdot}{\scalerel*{\cdot}{\bigodot}}
\DeclareMathSymbol{\shortminus}{\mathbin}{AMSa}{"39}

\makeatletter
\DeclareTextCompositeCommand{\r}{OT1}{A}{%
	\leavevmode\vbox{%
		\offinterlineskip
		\ialign{\hfil##\hfil\cr\char23\cr\noalign{\kern-1.15ex}A\cr}%
	}%
}
\makeatother

%opening
\title{Q-POP-FerroDyn}


\begin{document}
	
	\maketitle
	\section*{Formulation for Q-POP-FerroDyn}

		The total free energy of the system is defined as 
		\begin{equation}
			F = F_{Landau} + F_{gradient} + F_{electrostatic} + F_{elastic},
		\end{equation}
		where
		\begin{align}
			F_{Landau} & = \int(a_{i}P_{i}^{2} + a_{ij}P_{i}^{2}P_{j}^{2} + a_{ijk}P_{i}^{2}P_{j}^{2}P_{k}^{2})dx^{3} \\
			F_{gradient} & = \int g_{ijkl}\pdv[]{P_{i}}{x_{j}}\pdv[]{P_{k}}{x_{l}}dx^{3} \\
			F_{electrostatic} & = \int\bigg(-\frac{1}{2}\kappa_{0}\kappa_{ij}^{b}E_{i}E_{j} - E_{i}P_{i}\bigg)dx^{3}\\
			F_{elastic} & = \int\frac{1}{2}c_{ijkl}(\epsilon_{ij}-\epsilon_{ij}^{0})(\epsilon_{kl}-\epsilon_{kl}^{0})dx^{3}
		\end{align}

		Dynamical equation for polarization \textbf{P}(x, t):
		\begin{equation}
			\mu_{ij}\pdv[2]{P_{j}}{t} + \gamma_{ij}\pdv[]{P_{j}}{t} + \frac{\delta F}{\delta P_{i}} = 0,
		\end{equation}
		where $\bm{\mu}$ and $\bm{\gamma}$ are mass and damping coefficients, respectively. 
		

		The driving force term in the dynamic equation for polarization is
		\begin{align}
			\frac{\delta F}{\delta P_{i}} & = \frac{\delta}{\delta P_{j}}(F_{Landau} + F_{gradient} + F_{electrostatic} + F_{elastic}) \\
			& = 2a_{i}P_{i} + 2a_{ij}P_{i}P_{j}^{2} + 2a_{ijk}P_{i}P_{j}^{2}P_{k}^{2}\ +\nonumber \\
			& \ \ \ \ g_{ijkl}\frac{\partial^{2}P_{k}}{\partial x_{j}\partial x_{l}} - E_{i}
		\end{align}

		Elastodynamics equation for mechanical displacement \textbf{u}(x, t):
		\begin{equation}
			\rho\pdv[2]{u_{i}}{t} = f^{v}_{i} + \frac{\partial}{\partial x_{j}}\bigg(\sigma_{ij} + \beta\pdv[]{\sigma_{ij}}{t}\bigg),
		\end{equation}
		where $\rho$, $\beta$, and \textbf{$f^{v}$} are the material mass density, the stiffness damping coefficient, and the external body force density, respectively. 
		
		The stress field $\bm{\sigma}(x, t)$ is defined as
		\begin{equation}
			\bm{\sigma}(\bm{x}, t) \equiv \sigma_{ij} = c_{ijkl}(\epsilon_{kl} - \epsilon_{kl}^{0}),
		\end{equation}
		where $\bm{c}$ is the elastic stiffness tensor, $\bm{\epsilon}(\textbf{x}, t)$ is the strain field given by $\bm{\epsilon} = [\nabla\bm{u}+(\nabla\bm{u})^{T}]/2$, and $\bm{\epsilon}^{0}(\bm{x}, t)\equiv \epsilon_{ij}^{0} = Q_{ijkl}P_{k}P_{l}$ is the eigenstrain field with an electrostrictive coefficient $\bm{Q}$. 
		\newpage 
		
		\section*{Maxwell's Equations}
		Starting from Ampere's Law,
		\begin{align} \label{eq:ampere}
			\nabla\cross\bm{H} &= \bm{J} + \pdv{\bm{D}}{t} \nonumber\\
			&= \bm{J} + \epsilon\pdv{\bm{E}}{t} + \pdv{\bm{P}}{t}
		\end{align}
		\begin{align} \label{eq:ampere2}
			\implies \pdv{\bm{E}}{t} &= \frac{1}{\epsilon}\bigg(\nabla\cross\bm{H} - \bm{J} - \pdv{\bm{P}}{t}\bigg) \nonumber \\
				&= \frac{1}{\epsilon_{0}\epsilon_{r}}\bigg(\nabla\cross\bm{H} - \bm{J}^{f}-\pdv{\bm{P}}{t}\bigg)
		\end{align}
		
		Faraday's law reads
		\begin{equation} \label{eq:faraday}
			\nabla\cross\bm{E} = -\mu\bigg(\pdv{\bm{M}}{t} + \pdv{\bm{H}}{t}\bigg)
		\end{equation}
		\begin{equation} \label{eq:faraday2}
			\implies\pdv{\bm{H}}{t} = -\frac{1}{\mu}(\nabla\cross\bm{E}) - \pdv{\bm{M}}{t}
		\end{equation}
		
		The Maxwell's equations are solved using the finite-difference time-domain method (FDTD), using a staggered Yee grid. The stencil used by the Yee grid is as follows:
		\begin{align*}
			& 1: (i-1, j-1, k-1), 2: (i-1, j-1, k), 3: (i-1, j, k-1), 4: (i-1, j, k) \\
			& 5: (i, j-1, k-1), 6: (i, j-1, k), 7: (i, j, k-1), 8: (i, j, k)
		\end{align*}
		The permittivity and conductivity tensors are computed at each node by averaging over the stencil.
		\begin{align*}
			\epsilon_{ij} = \frac{\sum_{n = 1}^{8} \epsilon_{ij, n}}{8} \\
			\sigma_{ij} = \frac{\sum_{n = 1}^{8} \sigma_{ij, n}}{8} 
		\end{align*}
		Similarly, the source-terms representing various types of current are averaged over the same stencil.
		
		A fourth-order explicit Runge-Kutta (RK4) method is used for time-marching. 
		
		For n = 1, 2, 3, 4, where each n represents the respective strides between two formal time-steps, an example of the update equation at node (i, j, k) is as follows:
		\begin{align}
			\Delta E_{x}^{(t+\frac{n\Delta t}{4})}|_{i, j, k} = 
			& \bigg[\frac{\Delta t}{\epsilon_{0}}\bigg(\frac{\epsilon_{yy}\epsilon_{zz} - \epsilon_{yz}\epsilon_{zy}}{||\epsilon||}\bigg)\bigg|_{i, j, k} \times\dots\nonumber\\
			&\bigg(\left.\pdv{H_{z}}{y}\right\rvert^{(t + \frac{(n-1)\Delta t}{4})}_{i, j, k} - \left.\pdv{H_{y}}{z}\right\rvert^{(t + \frac{(n-1)\Delta t}{4})}_{i, j, k} - \Delta J_{x}^{(t + \frac{(n-1)\Delta t}{4})} - \sigma_{xi}E_{i, store}^{(t + \frac{(n-1)\Delta t}{4})} - \frac{\Delta P_{x}^{(t + \frac{n\Delta t}{4})}}{\Delta t}\bigg)\bigg]\nonumber \\
			& + \bigg[\frac{\Delta t}{\epsilon_{0}}\bigg(\frac{\epsilon_{xz}\epsilon_{zy} - \epsilon_{xy}\epsilon_{zz}}{||\epsilon||}\bigg)\bigg|_{i, j, k} \times\dots\nonumber\\
			&\bigg(\left.\pdv{H_{x}}{z}\right\rvert^{(t + \frac{(n-1)\Delta t}{4})}_{i, j, k} - \left.\pdv{H_{z}}{x}\right\rvert^{(t + \frac{(n-1)\Delta t}{4})}_{i, j, k} - \Delta J_{y}^{(t + \frac{(n-1)\Delta t}{4})} - \sigma_{yi}E_{i, store}^{(t + \frac{(n-1)\Delta t}{4})} - \frac{\Delta P_{y}^{(t + \frac{n\Delta t}{4})}}{\Delta t}\bigg)\bigg] \nonumber\\
			& + \bigg[\frac{\Delta t}{\epsilon_{0}}\bigg(\frac{\epsilon_{xy}\epsilon_{yz} - \epsilon_{xz}\epsilon_{yy}}{||\epsilon||}\bigg)\bigg|_{i, j, k} \times\dots\nonumber\\
			&\bigg(\left.\pdv{H_{y}}{x}\right\rvert^{(t + \frac{(n-1)\Delta t}{4})}_{i, j, k} - \left.\pdv{H_{x}}{y}\right\rvert^{(t + \frac{(n-1)\Delta t}{4})}_{i, j, k} - \Delta J_{z}^{(t + \frac{(n-1)\Delta t}{4})} - \sigma_{zi}E_{i, store}^{(t + \frac{(n-1)\Delta t}{4})} - \frac{\Delta P_{z}^{(t + \frac{n\Delta t}{4})}}{\Delta t}\bigg)\bigg]
		\end{align}
		where
		\begin{equation}
			\left.\pdv{H_{x}}{y}\right\rvert^{(t + \frac{(n-1)\Delta t}{4})} = \frac{H_{x}^{(t + \frac{(n-1)\Delta t}{4})}|_{i, j, k} - H_{x}^{(t + \frac{(n-1)\Delta t}{4})}|_{i-1, j, k}}{\Delta y},
		\end{equation}

		\begin{equation}
			\Delta J_{i} = \Delta J_{f, i} + \Delta J_{p, i} + \Delta J_{ISHE, i},
		\end{equation}

		$\Delta J_{f, i}$ is the change in the free charge current density. \\

		$\Delta J_{p, i}$ is the change in the polarization current density (eddy current). Its evolution is governed by the Drude equation 
		\begin{equation}
			\pdv{\bm{J}_{p}}{t} + \frac{\bm{J}_{p}}{\tau} = \epsilon_{0}\omega_{p}^{2}\bm{E} 
		\end{equation}
		
		For n = 1, 2, 3, 4:
		\begin{align}
			\Delta J_{p, i}^{(t+\frac{n\Delta t}{4})} &= 
				\Delta J_{p, i, n1}^{(t+\frac{n\Delta t}{4})} + \Delta J_{p, i, n2}^{(t+\frac{n\Delta t}{4})} + \Delta J_{p, i, n3}^{(t+\frac{n\Delta t}{4})} \nonumber\\
				&= \epsilon_{0}E_{i, cell}^{t}(\omega_{p, n1}\omega_{p, n1}comp_{n1} + \omega_{p, n2}\omega_{p, n2}comp_{n2} + \omega_{p, n3}\omega_{p, n3}comp_{n3}) - \dots \nonumber\\
				& \Delta t\bigg(\frac{J_{p, i, n1}^{(t + \frac{(n-1)\Delta t}{4})}}{\tau_{e, n1}} + \frac{J_{p, i, n2}^{(t + \frac{(n-1)\Delta t}{4})}}{\tau_{e, n2}} + \frac{J_{p, i, n3}^{(t + \frac{(n-1)\Delta t}{4})}}{\tau_{e, n3}}\bigg),
		\end{align}
		where
		\begin{align}
			J_{p, i, store}^{(t + \frac{(n-1)\Delta t}{4})} = 
			\begin{cases}
				J_{p, i}^{(t)}, & n = 1\\
				J_{p, i}^{(t)} + 0.5*\Delta J_{p, i}^{(t+\frac{(n-1)\Delta t}{4})}, & n = 2, 3\\
				J_{p, i}^{(t)} + \Delta J_{p, i}^{(t+\frac{3\Delta t}{4})}, & n = 4,
			\end{cases} 
		\end{align} \\
		
		$\Delta J_{ISHE, i}$ is the change in the current density induced by the spin current via the inverse spin-Hall effect. 
		\begin{align}
			\Delta J_{ISHE, i} = \dots
		\end{align}
		
		
		\begin{align}
			E_{i, store}^{(t + \frac{(n-1)\Delta t}{4})} = 
			\begin{cases}
				E_{i}^{(t)}, & n = 1\\
				E_{i}^{(t)} + 0.5*\Delta E_{i}^{(t+\frac{(n-1)\Delta t}{4})}, & n = 2, 3\\
				E_{i}^{(t)} + \Delta E_{i}^{(t+\frac{3\Delta t}{4})}, & n = 4
			\end{cases} 
		\end{align}
		
		Finally, the polarization current is calculated by evolving the polarization dynamical equation using RK4 time-stepping.
	
		\begin{align}
			\Delta H_{x}^{(t+\frac{n\Delta t}{4})}|_{i, j, k} = -\frac{\Delta t}{\mu_{0}}\bigg(\pdv{E_{z}}{y} - \pdv{E_{y}}{z}\bigg) - \frac{\Delta M_{x, 1} + \Delta M_{x, 2}}{M_{\text{count}}}
		\end{align}
		
		\begin{align}
			E_{x}^{t + \Delta t} = E_{x}^{t} + \frac{\Delta E_{x}^{(t+\frac{\Delta t}{4})}}{6} + \frac{\Delta E_{x}^{(t+\frac{2\Delta t}{4})}}{3} + \frac{\Delta E_{x}^{(t+\frac{3\Delta t}{4})}}{3} + \frac{\Delta E_{x}^{(t+\frac{4\Delta t}{4})}}{6} \\
			H_{x}^{t + \Delta t} = H_{x}^{t} + \frac{\Delta H_{x}^{(t+\frac{\Delta t}{4})}}{6} + \frac{\Delta H_{x}^{(t+\frac{2\Delta t}{4})}}{3} + \frac{\Delta H_{x}^{(t+\frac{3\Delta t}{4})}}{3} + \frac{\Delta H_{x}^{(t+\frac{4\Delta t}{4})}}{6}
		\end{align}
		\newpage
		\section*{PML Modification}
		The time-harmonic forms of the Maxwell's equations \eqref{eq:ampere} and \eqref{eq:faraday} are
		\begin{align}
			\nabla\cross\bm{\widehat{H}} &= \bm{\widehat{J}} + \textit{i}\omega\epsilon_{0}\epsilon_{r}\bm{\widehat{E}} + \textit{i}\omega\bm{\widehat{P}} \\
			\nabla\cross\bm{\widehat{E}} &= -\textit{i}\omega\mu_{0}\mu_{r}(\bm{\widehat{M}+\widehat{H}})
		\end{align}
		where $\bm{\widehat{E}}, \bm{\widehat{H}}, \bm{\widehat{J}}, \bm{\widehat{M}}, \text{and} \bm{\widehat{P}}$ are all in the frequency-domain.
		
		A UPML constitutive tensor is now defined,
		\begin{align}
			\bar{\bar{s}} &= \begin{bmatrix}s_{x}^{\shortminus 1} & 0 & 0 \\ 0 & s_{x} & 0 \\ 0 & 0 & s_{x}\end{bmatrix}\begin{bmatrix}s_{y} & 0 & 0 \\ 0 & s_{y}^{\shortminus 1} & 0 \\ 0 & 0 & s_{y}\end{bmatrix}\begin{bmatrix}s_{z} & 0 & 0 \\ 0 & s_{z} & 0 \\ 0 & 0 & s_{z}^{\shortminus 1}\end{bmatrix} \\
			&= \begin{bmatrix}s_{x}^{\shortminus 1}s_{y}s_{z} & 0 & 0 \\ 0 & s_{x}s_{y}^{\shortminus 1}s_{z} & 0 \\ 0 & 0 & s_{x}s_{y}s_{z}^{\shortminus 1}\end{bmatrix}
		\end{align}
		where 
		\begin{equation}
			s_{x} = \kappa_{x} + \frac{\sigma_{x}}{i\omega\epsilon};\;\;\; s_{y} = \kappa_{y} + \frac{\sigma_{y}}{i\omega\epsilon};\;\;\; s_{z} = \kappa_{z} + \frac{\sigma_{z}}{i\omega\epsilon}
		\end{equation}
		
		$\kappa$ and $\sigma$ are all one-dimensional functions that are equal to unity and zero respectively in the physical non-PML region, ensuring that the PML constitutive tensor and PML auxilliary equations are active only in the PML.
		
		There are no sources of current, polarization and magnetization densities within the PML, consequently reducing the time-harmonic equations to simply
		\begin{align}
			\nabla\cross\bm{\widehat{H}} &= \textit{i}\omega\bar{\bar{s}}\epsilon_{0}\epsilon_{r}\bm{\widehat{E}} \\
			\nabla\cross\bm{\widehat{E}} &= -\textit{i}\omega\bar{\bar{s}}\mu_{0}\mu_{r}\widehat{H}
		\end{align}
		
		To decouple the frequency-dependent terms and thereby avoid a convolution between the material tensors and the EM-fields in the time-domain, new constitutive relationships are defined as follows:
		\begin{align} \label{eq:const}
			\bm{\widehat{D_{x}}} = \epsilon_{0}\epsilon_{r}\frac{s_{z}}{s_{x}}\bm{\widehat{E_{x}}} \\
			\bm{\widehat{B_{x}}} = \epsilon_{0}\epsilon_{r}\frac{s_{z}}{s_{x}}\bm{\widehat{H_{x}}} \\
		\end{align}
		
		Following an inverse Fourier transform, the time-domain differential equations are now
		\begin{equation}
			\begin{bmatrix}
				\pdv{H_{z}}{y} - \pdv{H_{y}}{z} \\[4pt]
				\pdv{H_{x}}{z} - \pdv{H_{z}}{x} \\[4pt]
				\dv{H_{y}}{x} - \pdv{H_{x}}{y}
			\end{bmatrix}
			 = \pdv{}{t}
			 \begin{bmatrix}
			 	\kappa_{y} & 0 & 0 \\[4pt]
			 	0 & \kappa_{z} & 0 \\[4pt] 
			 	0 & 0 & \kappa_{x}
			 \end{bmatrix}
			 \begin{bmatrix}
			 	D_{x}\\[4pt]
			 	D_{y}\\[4pt]
			 	D_{z}
			 \end{bmatrix}
			 + \frac{1}{\epsilon_{0}}
			 \begin{bmatrix}
			 	\sigma_{y} & 0 & 0 \\[4pt]
			 	0 & \sigma_{z} & 0 \\[4pt] 
			 	0 & 0 & \sigma_{x}
			 \end{bmatrix}
			 \begin{bmatrix}
			 	D_{x}\\[4pt]
			 	D_{y}\\[4pt]
			 	D_{z}
			 \end{bmatrix}
		\end{equation}
		Taking for instance the update equation for $D_{x}$,
		\begin{equation}
			\pdv{D_{x}}{t} = \frac{\bigg(\pdv{H_{z}}{y} - \pdv{H_{y}}{z}\bigg)}{\kappa_{y}} - \frac{\sigma_{x}D_{x}}{\kappa_{y}\epsilon_{0}}
		\end{equation}
		A similar RK4 treatment as earlier yields 
		\begin{align*}
			\Delta D_{x}^{(t+\frac{n\Delta t}{4})}|_{i, j, k} = 
			& \frac{\Delta t}{\kappa_{y}}\bigg(\pdv{H_{z}}{y}\bigg|^{(t + \frac{(n-1)\Delta t}{4})}_{i, j, k} - \pdv{H_{y}}{z}\bigg|^{(t + \frac{(n-1)\Delta t}{4})}_{i, j, k}\bigg) - \frac{\sigma_{y}\Delta t}{\kappa_{y}\epsilon_{0}}D_{x, store}^{(t + \frac{(n-1)\Delta t}{4})}\\
		\end{align*}
		\begin{equation}
			D_{x}^{t + \Delta t} = D_{x}^{t} + \frac{\Delta D_{x}^{(t+\frac{\Delta t}{4})}}{6} + \frac{\Delta D_{x}^{(t+\frac{2\Delta t}{4})}}{3} + \frac{\Delta D_{x}^{(t+\frac{3\Delta t}{4})}}{3} + \frac{\Delta D_{x}^{(t+\frac{4\Delta t}{4})}}{6}
		\end{equation}
		
		Multiplying \eqref{eq:const} by $i\omega$ and performing an inverse Fourier transform returns the new value of the electric field.
		\begin{align}
			\pdv{}{t}\begin{bmatrix}
				\kappa_{x} & 0 & 0 \\[4pt]
				0 & \kappa_{y} & 0 \\[4pt] 
				0 & 0 & \kappa_{z}
			\end{bmatrix}
			\begin{bmatrix}
				D_{x} \\[4pt]
				D_{y} \\[4pt]
				D_{z} 
			\end{bmatrix} + 
			\frac{1}{\epsilon_{0}}\begin{bmatrix}
				\sigma_{x} & 0 & 0 \\[4pt]
				0 & \sigma_{y} & 0 \\[4pt] 
				0 & 0 & \sigma_{z}
			\end{bmatrix}
			\begin{bmatrix}
			D_{x} \\[4pt]
			D_{y} \\[4pt]
			D_{z} 
			\end{bmatrix} = &
			\pdv{}{t}
			\begin{bmatrix}
				\epsilon_{xx} & \epsilon_{xy} & \epsilon_{xz} \\[4pt]
				\epsilon_{yx} & \epsilon_{yy} & \epsilon_{yz} \\[4pt] 
				\epsilon_{zx} & \epsilon_{zy} & \epsilon_{zz}
			\end{bmatrix}
			\begin{bmatrix}
				\kappa_{z} & 0 & 0 \\[4pt]
				0 & \kappa_{x} & 0 \\[4pt] 
				0 & 0 & \kappa_{y}
			\end{bmatrix}
			\begin{bmatrix}
				E_{x} \\[4pt]
				E_{y} \\[4pt]
				E_{z} 
			\end{bmatrix} + \dots \\ &
			\frac{1}{\epsilon_{0}}			\begin{bmatrix}
				\epsilon_{xx} & \epsilon_{xy} & \epsilon_{xz} \\[4pt]
				\epsilon_{yx} & \epsilon_{yy} & \epsilon_{yz} \\[4pt] 
				\epsilon_{zx} & \epsilon_{zy} & \epsilon_{zz}
			\end{bmatrix}\begin{bmatrix}
				\sigma_{z} & 0 & 0 \\[4pt]
				0 & \sigma_{x} & 0 \\[4pt] 
				0 & 0 & \sigma_{y}
			\end{bmatrix}
			\begin{bmatrix}
				E_{x} \\[4pt]
				E_{y} \\[4pt]
				E_{z} 
			\end{bmatrix}
		\end{align}
		
		The update equation for E is then 
		\begin{align}
			\Delta E_{x}^{(t+\frac{n\Delta t}{4})}|_{i, j, k} = &
			\frac{\Delta t}{\kappa_{z}}\times\bigg(\frac{\epsilon_{yy}\epsilon_{zz} - \epsilon_{yz}\epsilon_{zy}}{||\epsilon||}\bigg)\bigg|_{i, j, k}\times\dots \nonumber \\
			&\bigg(\frac{\kappa_{x}\Delta D_{x}^{(t+\frac{n\Delta t}{4})}|_{i, j, k}}{\Delta t} + \frac{\sigma_{x}D_{x, store}^{(t + \frac{(n-1)\Delta t}{4})}}{\epsilon_{0}}\bigg) - \frac{\Delta t}{\kappa_{z}}\frac{\sigma_{z}E_{x, store}^{(t + \frac{(n-1)\Delta t}{4})}}{\epsilon_{0}}
		\end{align}
\end{document}
